\begin{description}
\item[Problem] Let $p \ge 2$. Prove that if $2^p - 1$ is prime, then $p$ must
also be prime.

\item[Proof] We need to prove first that $a - 1$ divides $a^n - 1$ for all $a,
n \in \mathbb Z$. Let's do it by induction. $a - 1$ obviously divides $a^1 - 1
= a - 1$, so the statement holds for the base. Now we need to prove that
$a^{n+1} - 1$ is divided by $a - 1$ if $a-1$ divides $a^n - 1$. But $a^{n+1}-1
= a \cdot a^n - 1 = a \cdot a^n - a^n + a^n - 1 = a^n (a - 1) + (a^n - 1)$.  By
the induction hypothesis, $a^n - 1 = k(a-1)$ for some $k \in \mathbb Z$, so $a
- 1 | a^n (a - 1) + (a^n - 1) \iff a - 1 | a^n (a-1) + k(a-1) \iff a - 1 | (a -
1) (a^n + k)$, which is true.

Next, let $p = nm$. Then $2^{nm} - 1 = \left(2^n\right)^m - 1$ is prime. But
then $2^n - 1$ must divide $2^p - 1$.  We know that $2^p - 1$ is a prime, so it
can only be divided by itself or $1$. If $2^n - 1 = 2^p - 1$, then $n = p$ and
$m = 1$. If $2^n - 1 = 1$, then $2^n = 2$, and $n = 1$, while $m = p$. Thus,
$p$ can't be a composite number.

\end{description}
