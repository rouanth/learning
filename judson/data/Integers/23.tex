\begin{description}

\item[Problem] Define the \textbf{\textit{least common multiple}} of two
nonzero integers $a$ and $b$, denoted by $\text{lcm}(a, b)$, to be the
non-negative integer $m$ such that both $a$ and $b$ divide $m$, and if $a$ and
$b$ divide any other integer $n$, them $m$ also divides $n$. Prove that any two
integers $a$ and $b$ have a unique least common multiple.

\item[Proof] Let $d = \gcd(a,b)$. Then $a = da'$, $b = db'$, where $a'$ and
$b'$ are relatively prime integers which are also both relatively prime with
$d$.

A common multiple of $a$ and $b$ is $|a'b'|d$. $a$ divides $|a'b'|d = |ab'|$,
and $b$ divides $|a'b'd| = |a'b|$. We need to prove that $|a'b'|d$ is the least
common divisor, that is, it divides each common divisor of $a$ and $b$. $a'$
and $d$ must divide every common divisor of $a$ and $b$ in order for it to be a
multiple of $a$, and likewise $b'$ and $d$ must divide it so $b$ would divide
it as well. But then, since $a'$, $b'$ and $d$ are pairwise relatively prime,
$|a'b'|d$ must divide each common multiple. Hence, $|a'b'|d$, also possibly
written as $\frac{|ab|}{\gcd a b}$, is the least common multiple.

\end{description}
