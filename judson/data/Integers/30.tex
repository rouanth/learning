\begin{description}
\item[Problem] Prove that there are an infinite number of primes of the form
$4n - 1$.

\item[Proof] Similarly to the previous exercise, by using the Division
Algorithm we arrive at conclusion that the prime numbers greater that $2$ can
only be of the forms $4n + 1$ or $4n + 3$. But $4n + 3$ is the same as $4(n+1)
-1$. If we replace $n+1$ in this notation with $n$, nothing changes, the same
numbers are present in the class defined by $4n-1$. So all we have to do is
prove that there are an infinite number of primes of the form $4n+3$.

We shall analyze the number $q = 3 + 4\prod_{i=3}^k p_i$ where $p_k$ is the
highest known prime number of the form $4n+3$. $q$ is not divided by any prime
number of the form $4n + 3$. It also can't be factored solely into prime
numbers of the form $4n + 1$, because multiplying them produced only the
numbers of the same form: $(4n_1+1)(4n_2+1) = 4 (4 n_1 n_2 + n_1 + n_2) + 1$.
Hence, $q$ must be the next prime number of the form $4n + 3$.

\end{description}
