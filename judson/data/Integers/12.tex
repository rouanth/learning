\begin{description}
\item[Problem] (Power Sets) Let $X$ be a set. Define the \textit{\textbf{power
set}} of $X$, denoted $\wp (X)$, to be the set of all subsets of $X$. For
example,
$$\wp\left(\{a, b\}\right) = \{\emptyset, \{a\}, \{b\}, \{a, b\}\}$$

For every positive integer $n$, show that a set with exactly $n$ elements has
a power set with exactly $2^n$ elements.

\item[Proof] If there is a single element $a$, the power set consists of $|\{
\emptyset, a\}| = 2 = 2^1$ elements.

If a set has $n + 1$ elements, then it has $n$ elements and some additional
element $b$. Then its power set contains all the elements from the power set
for $n$ elements and them once again, but this time with $b$ added to each.
Then there are two times as many elements in the power set for $n + 1$ elements
as there are in the power set for $n$ elements.

\end{description}
