\begin{description}
\item[Problem] Show that the Principle of Well-Ordering for the natural numbers
implies that 1 is the smallest natural number. Use this result to show that the
Principle of Well-Ordering implies the Principle of Mathematical Induction;
that is, show that if $S \subset \mathbb{N}$ such that $1 \in S$ and $n + 1 \in
S$ whenever $n \in S$, then $S = \mathbb{N}$.

\item[Proof] First, due to the Principle of Well-Ordering, natural numbers,
being a subset of themselves, have a smallest element. If it's not 1, then it
can be expressed as $n + 1$, with $n$ being less than the element in question,
which contradicts our assumption. So 1 must be the smallest natural number.

Next, we need to prove that if $1 \in S$ and $n + 1 \in S$ whenever $n \in S$,
then $S = \mathbb{N}$. Let's assume the contrary: even if all the conditions
are met, there exists a set $A$ of natural numbers that are not in $S$. Then
there is a least element in $A$, let's call it $a$. $a$ can't be $1$ since $1$
is known to be in $S$. Then it can be expressed as $n + 1$. But since $n \in
S$, $n + 1$ must also be in $S$. We arrive at a contradiction, proving the
Principle of Mathematical Induction.

\end{description}
