\begin{description}
\item[Problem] Prove that the two principles of mathematical induction are
equivalent.

\item[Proof] The two principles are equivalent if each one implies the other.

Let's show that the First Principle implies the Second. First, it implies the
Principle of Well-Ordering, and we'll use that. We shall assume that the Second
Principle of Mathematical Induction is wrong~--- that is, if $S(n_0)$ is true
for some $n_0$ and if $S(n_0), S(n_0 + 1), \dots, S(k)$ imply $S(k+1)$ for $k
\ge n_0$, there still exist integers of set $A$ that are greater than $n_0$ for
which $S$ isn't true. By the Principle of Well-Ordering, there is a least
integer in $A$, let's call it $a$. Then $S(n_0), S(n_0 + 1), \dots, S(a - 1)$
are true: none of the integers $n_0, n_0 + 1, \dots, a - 1$ can be in $A$ since
they are less then $a$. But $S(n_0), S(n_0 + 1), \dots, S(a - 1)$ imply $S(a)$,
so $S(a)$ is also true. We arrive at a contradiction, thus proving the Second
Principle of Mathematical Induction with the First.

Proving the reverse is more simple. We need to prove that if $S(k)$ implies
$S(k+1)$ for $k \ge n_0$ and $S(n_0)$ is true, then $S(n)$ is true for $n \ge
n_0$. We know that if $S(n_0)$ is true, then it is sufficient to prove that
$S(n_0), S(n_0 + 1), \dots, S(k)$ imply $S(k+1)$ in order to prove $S(n)$ for
$n \ge n_0$. But proving this is trivial since even just $S(k)$ implies
$S(k+1)$, and so the stronger statement~--- that $S(n_0), S(n_0 + 1), \dots,
S(k)$ imply $S(k+1)$~--- holds as well.

\end{description}
